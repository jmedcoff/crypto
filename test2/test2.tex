\documentclass{amsart}
\usepackage{amssymb}
\usepackage{amsmath}
\usepackage{geometry}
\geometry{a4paper}

\title{Test 2}
\author{Jason Medcoff}
\date{}

\begin{document}
	\maketitle
	
	\noindent{\textbf{1.}}
	Consider the curve $Y^2 = X^3 - 2X + 4$ and the points $P = (0, 2)$, $Q = (3, -5)$. Compute,
	by hand, $P \oplus Q$.
	
	\textit{Solution.}
	First, we obtain the equation of the line passing through $P$ and $Q$ as $y = -\frac{7}{3}x + 2$. Next, we want the intersections of this line with the curve, so set
	\begin{equation*}
	\begin{split}
	\bigg(-\frac{7}{3}x + 2\bigg)^2 &= x^3 - 2x + 4 \\
	0 &= x^3 - \frac{49}{9}x^2 + \frac{22}{3}x
	\end{split}
	\end{equation*}
	Clearly, $x=0$ is a solution, so we divide by $x$.
	$$ 0 = x^2 - \frac{49}{9}x + \frac{22}{3} $$
	As we already know from point $Q$, $x=3$ is also a solution so we use polynomial long division to obtain $x = \frac{22}{9}$. Then from the equation of the curve,
	$$ y^2 = (\frac{22}{9})^3 - 2(\frac{22}{9}) + 4 $$
	and we have that 
	$$ y = \pm \sqrt{\frac{10000}{729}} = \pm \frac{100}{27} . $$
	Plugging our $x = \frac{22}{9}$ into the line equation, we have the negative value, so the positive value gives the correct $y$ for the new point.
	So the result of $P \oplus Q$ is $(\frac{22}{9}, \frac{100}{27})$.
	
	$\newline$
	\noindent{\textbf{2.}}
	Consider the cubic polynomial
	$$ x^3 + ax + b = (x - e_1)(x-e_2)(x-e_3) $$
	and show that $4a^3 + 27b^2 = 0$ if and only if two or more of $e_1$, $e_2$, $e_3$ are the same.
	
	\textit{Solution.}
	Recall from elementary algebra that the discriminant is a function of a polynomial's coefficients, that gives interesting properties of the polynomial. Namely the discriminant equals zero if and only if the polynomial has a multiple root. So, we will begin by finding the discriminant of the given polynomial.
	
	If we expand $(x - e_1)(x-e_2)(x-e_3)$, we have
	$$ x^3 - x^2(e_1 + e_2 + e_3) + x(e_1e_2 + e_1e_3 + e_2 e_3) - (e_1e_2e_3) $$
	but clearly it must be that $(e_1 + e_2 + e_3) = 0$. Then make the substitution $e_3 = -e_1 - e_2$. Then we obtain
	\begin{equation}
	x^3 + x(-e_1^2 - e_1e_2 - e_2^2) + (e_1^2e_2 + e_1e_2^2)
	\end{equation}
	and it is plainly clear how $a$ and $b$ are expressed in terms of the roots.
	
	The discriminant for a cubic with roots $e_1$, $e_2$, $e_3$ is given by
	$$ \Delta_f := \prod_{i\neq j} (e_i - e_j) = (e_1 - e_2)(e_2 - e_1)(e_3 - e_2)(e_2 - e_3)(e_1 - e_3)(e_3 - e_1) $$
	
	and since $e_3 = - e_1 - e_2$, we can make the substitution
	$$ \Delta_f = (e_1 - e_2)(e_2 - e_1)((-e_1-e_2) - e_2)(e_2 - (-e_1-e_2))(e_1 - (-e_1-e_2))((-e_1-e_2) - e_1) . $$
	
	Expanding yields
	\begin{equation}
	\Delta_f = -4 e_1^6 - 12 e_1^5 e_2 + 3 e_1^4 e_2^2 + 26 e_1^3 e_2^3 + 3 e_1^2 e_2^4 - 12 e_1 e_2^5 - 4 e_2^6 .
	\end{equation}
	
	If we take $4a^3 + 27b^2$ and make the substitution for $a$ and $b$ shown in equation (1) in terms of the roots, we have
	\begin{equation*}
	\begin{split}
	4 p^3 + 27 q^2 & = 4 (-e_1^2 - e_1e_2 - e_2^2)^3 + 27 (e_1^2e_2 + e_1e_2^2)^2 \\
	&= -4 e_1^6 - 12 e_1^5 e_2 + 3 e_1^4 e_2^2 + 26 e_1^3 e_2^3 + 3 e_1^2 e_2^4 - 12 e_1 e_2^5 - 4 e_2^6
	\end{split}
	\end{equation*}
	which is precisely equation (2). Thus $4a^3 + 27b^2$ is the discriminant and it follows that $4a^3 + 27b^2 = 0$ if and only if two or more of $e_1$, $e_2$, $e_3$ are the same.
	
	
	
	
	
	
	
	
	
	
	
	
	
	
	
	
	
	
	
	
	
	
	
	
	
\end{document}