\documentclass{amsart}
\usepackage{amssymb}
\usepackage{amsmath}
\usepackage{enumitem}
\usepackage{listings}
\usepackage{geometry}
\geometry{a4paper}

\title{Test 2}
\author{Jason Medcoff}
\date{}

\begin{document}
	\maketitle
	
	\noindent{\textbf{1.}}
	Consider the curve $Y^2 = X^3 - 2X + 4$ and the points $P = (0, 2)$, $Q = (3, -5)$. Compute,
	by hand, $P \oplus Q$.
	
	\textit{Solution.}
	First, we obtain the equation of the line passing through $P$ and $Q$ as $y = -\frac{7}{3}x + 2$. Next, we want the intersections of this line with the curve, so set
	\begin{equation*}
	\begin{split}
	\bigg(-\frac{7}{3}x + 2\bigg)^2 &= x^3 - 2x + 4 \\
	0 &= x^3 - \frac{49}{9}x^2 + \frac{22}{3}x
	\end{split}
	\end{equation*}
	Clearly, $x=0$ is a solution, so we divide by $x$.
	$$ 0 = x^2 - \frac{49}{9}x + \frac{22}{3} $$
	As we already know from point $Q$, $x=3$ is also a solution so we use polynomial long division to obtain $x = \frac{22}{9}$. Then from the equation of the curve,
	$$ y^2 = (\frac{22}{9})^3 - 2(\frac{22}{9}) + 4 $$
	and we have that 
	$$ y = \pm \sqrt{\frac{10000}{729}} = \pm \frac{100}{27} . $$
	Plugging our $x = \frac{22}{9}$ into the line equation, we have the negative value, so the positive value gives the correct $y$ for the new point.
	So the result of $P \oplus Q$ is $(\frac{22}{9}, \frac{100}{27})$.
	
	$\newline$
	\noindent{\textbf{2.}}
	Consider the cubic polynomial
	$$ x^3 + ax + b = (x - e_1)(x-e_2)(x-e_3) $$
	and show that $4a^3 + 27b^2 = 0$ if and only if two or more of $e_1$, $e_2$, $e_3$ are the same.
	
	\textit{Solution.}
	Recall from elementary algebra that the discriminant is a function of a polynomial's coefficients, that gives interesting properties of the polynomial. Namely the discriminant equals zero if and only if the polynomial has a multiple root. So, we will begin by finding the discriminant of the given polynomial.
	
	If we expand $(x - e_1)(x-e_2)(x-e_3)$, we have
	$$ x^3 - x^2(e_1 + e_2 + e_3) + x(e_1e_2 + e_1e_3 + e_2 e_3) - (e_1e_2e_3) $$
	but clearly it must be that $(e_1 + e_2 + e_3) = 0$. Then make the substitution $e_3 = -e_1 - e_2$. Then we obtain
	\begin{equation}
	x^3 + x(-e_1^2 - e_1e_2 - e_2^2) + (e_1^2e_2 + e_1e_2^2)
	\end{equation}
	and it is plainly clear how $a$ and $b$ are expressed in terms of the roots.
	
	The discriminant for a cubic with roots $e_1$, $e_2$, $e_3$ is given by
	$$ \Delta_f := \prod_{i\neq j} (e_i - e_j) = (e_1 - e_2)(e_2 - e_1)(e_3 - e_2)(e_2 - e_3)(e_1 - e_3)(e_3 - e_1) $$
	
	and since $e_3 = - e_1 - e_2$, we can make the substitution
	$$ \Delta_f = (e_1 - e_2)(e_2 - e_1)((-e_1-e_2) - e_2)(e_2 - (-e_1-e_2))(e_1 - (-e_1-e_2))((-e_1-e_2) - e_1) . $$
	
	Expanding yields
	\begin{equation}
	\Delta_f = -4 e_1^6 - 12 e_1^5 e_2 + 3 e_1^4 e_2^2 + 26 e_1^3 e_2^3 + 3 e_1^2 e_2^4 - 12 e_1 e_2^5 - 4 e_2^6 .
	\end{equation}
	
	If we take $4a^3 + 27b^2$ and make the substitution for $a$ and $b$ shown in equation (1) in terms of the roots, we have
	\begin{equation*}
	\begin{split}
	4 a^3 + 27 b^2 & = 4 (-e_1^2 - e_1e_2 - e_2^2)^3 + 27 (e_1^2e_2 + e_1e_2^2)^2 \\
	&= -4 e_1^6 - 12 e_1^5 e_2 + 3 e_1^4 e_2^2 + 26 e_1^3 e_2^3 + 3 e_1^2 e_2^4 - 12 e_1 e_2^5 - 4 e_2^6
	\end{split}
	\end{equation*}
	which is precisely equation (2). Thus $4a^3 + 27b^2$ is the discriminant and it follows that $4a^3 + 27b^2 = 0$ if and only if two or more of $e_1$, $e_2$, $e_3$ are the same.
	
	$\newline$
	\noindent{\textbf{3.}}
	Consider $$y^2 = x^3 +2x + 3 \ \text{over } \mathbb{F}_7$$
	\begin{enumerate}[label=(\alph*)]
		\item List all the points on this curve.
		
		\textit{Solution.}
		The points on the curve are those that satisfy the equation in $\mathbb{F}_7$. That is, we can calculate $x^3 +2x + 3$ for all $x$, then check if that number exists as a square in $\mathbb{F}_7$. Brute forcing, we have $y^2 \in \{3, 6, 1, 5, 0\}$ for $x \in \{0, 1, \dots, 6\}$. In addition, we have $a^2 \in \{0, 1, 2, 4\}$ for $a \in \{0, 1, \dots, 6\}$. Then the feasible $y$ values are 0 and 1. Specifically, we have $(6, 0)$, $(2, \pm1)$, and $(3, \pm1)$.
		
		
		\item Make an addition table for these points.
		
		\textit{Solution.}
		\begin{table}[h]
			\centering
			\caption{Addition table}
			\label{my-label}
			\begin{tabular}{r|ccccc}
				$\oplus$ & $(6,0)$       & $(2,1)$       & $(2,-1)$      & $(3,1)$       & $(3,-1)$      \\ \hline
				$(6,0)$  & $\mathcal{O}$ & $(3,1)$       & $(3,-1)$      & $(2,1)$       & $(2,-1)$      \\
				$(2,1)$  & $(3,1)$       & $(3,-1)$      & $\mathcal{O}$ & $(2,-1)$      & $(6,0)$       \\
				$(2,-1)$ & $(3,-1)$      & $\mathcal{O}$ & $(3,1)$       & $(6,0)$       & $(2,1)$       \\
				$(3,1)$  & $(2,1)$       & $(2,-1)$      & $(6,0)$       & $(3,-1)$      & $\mathcal{O}$ \\
				$(3,-1)$ & $(2,-1)$      & $(6,0)$       & $(2,1)$       & $\mathcal{O}$ & $(3,1)$      
			\end{tabular}
		\end{table}
	\end{enumerate}
	
	$\newline$
	\noindent{\textbf{4.}}
	Consider the curve $y^2 = x^3 + x + 1$ over $\mathbb{F}_5$ and the points $P = (4,2)$, $Q = (0,1)$. Solve the discrete log problem, i.e. find $n$ such that $Q = nP$.
	
	\textit{Solution.}
	Brute force all multiples of $P$ using code. Clearly $n=5$.
	\begin{table}[h]
		\centering
		\caption{Multiplication table}
		\label{boi}
		\begin{tabular}{r|ccccccccc}
			$n$  & 0             & 1        & 2        & 3        & 4        & 5        & 6        & 7        & 8        \\ \hline
			$nP$ & $\mathcal{O}$ & $(4, 2)$ & $(3, 4)$ & $(2, 4)$ & $(0, 4)$ & $(0, 1)$ & $(2, 1)$ & $(3, 1)$ & $(4, 3)$
		\end{tabular}
	\end{table}
	
	$\newline$
	\noindent{\textbf{5.}}
	Consider the group of points on $E$. Since $E$ is a finite group, every point in $E$ has finite order. Then for some $P \in E$, if $s$ is the smallest solution to $sP = \mathcal{O}$, $s$ is the order of $P$. We know then that $(is)P = (i)(sP) = (i)\mathcal{O} = \mathcal{O}$ for all $i$. Then consider $nP = Q$. By the division algorithm and laws of exponents for groups,
	\begin{equation*}
	\begin{split}
	nP &= Q \\
	(is + r)P &= Q \\
	(is)P \oplus (r)P &= Q \\
	\mathcal{O} \oplus (r)P &= Q \\
	(r)P &= Q
	\end{split}
	\end{equation*}
	but since $r$ is less than $s$, it must be the smallest such solution to the equation; therefore $r=n_0$.
	
	$\newline$
	\noindent{\textbf{6.}}
	Use the double-and-add algorithm to compute $nP$ in
	$$ y^2 = x^3 + 1828x + 1675 \ \text{over } \mathbb{F}_1999 $$
	for $n=11$ and $P = (1756, 348)$. List the intermediate steps in a table.
	
	\textit{Solution.}
	We obtain the answer as $(1068,1540)$. Table \ref{pow} displays the state of the intermediate point \texttt{r} in the code.	
	\begin{table}[h]
		\centering
		\caption{Exponentiation of $P$}
		\label{pow}
		\begin{tabular}{l|l}
			$n$ & $R$           \\ \hline
			11  & $(1756, 348)$ \\
			5   & $(1756, 348)$ \\
			2   & $(1362,998)$  \\
			1   & $(1362,998)$  \\
			0   & $(1068,1540)$
		\end{tabular}
	\end{table}
	
	
	\noindent{\textbf{7.}}
	Attach a listing of \textbf{your} code for
	\begin{enumerate}[label=(\alph*)]
		\item Addition of points on an elliptic curve over a finite field.
		\item Exponentiation $(nP)$.
	\end{enumerate}

	\textit{Solution.}
	Both pieces of code assume a point to be a 2-tuple. A curve is also a 2-tuple, given as $(a, b)$ for $y^2 = x^3 + ax + b$.
	
	\begin{lstlisting}[language=Python]
    def add(p1, p2, curve, p):
        if p1 is None:
            return p2
        if p2 is None:
            return p1
        if (p1[0] == p2[0]) and (p1[1] == (-p2[1]%p)):
            return None
        lam = 0
        if p1 == p2:
            lam = ((3*(p1[0]**2)+curve[0])*inverse_modp(2*p1[1], p)) % p
        else:
            lam = ((p2[1] - p1[1])*inverse_modp(p2[0] - p1[0], p)) % p
        x3 = (lam**2 - p1[0] - p2[0]) % p
        y3 = (lam*(p1[0] - x3) - p1[1]) % p
        return x3, y3
	\end{lstlisting}
	
	Note that the function \texttt{inverse\_modp} finds the multiplicative inverse of its argument mod \texttt{p}; it is also homemade, from a past homework.
	
	\begin{lstlisting}[language=Python]
    def ec_exp(pt, n, curve, p):
        q = pt
        r = None
        while n>0:
            if (n % 2) == 1:
                r = add(r, q, curve, p)
            q = add(q, q, curve, p)
            n = n>>1
        return r
	\end{lstlisting}
	
	
	
	
	
	
	
	
	
	
	
	
	
\end{document}