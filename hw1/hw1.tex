% Created 2018-01-08 Mon 14:00
\documentclass[11pt]{article}
\usepackage[utf8]{inputenc}
\usepackage[T1]{fontenc}
\usepackage{fixltx2e}
\usepackage{graphicx}
\usepackage{longtable}
\usepackage{float}
\usepackage{wrapfig}
\usepackage{rotating}
\usepackage[normalem]{ulem}
\usepackage{amsmath}
\usepackage{textcomp}
\usepackage{marvosym}
\usepackage{wasysym}
\usepackage{amssymb}
\usepackage{hyperref}
\tolerance=1000
\usepackage{geometry}
\geometry{a4paper}
\author{Jason Medcoff}
\date{8 January 2018}
\title{Simple Cipher Decryption}
\hypersetup{
  pdfkeywords={},
  pdfsubject={},
  pdfcreator={Emacs 25.1.1 (Org mode 8.2.10)}}
\begin{document}

\maketitle

\section{Frequency Analysis on Single Characters}
\label{sec-1}

My task was to decrypt the file \verb~HW3.txt.e~, developing necessary tools
along the way. I began by assuming the simplest case; the file was a
monoalphabetic cipher. Upon first examination of the file, I noticed
repetition of several one, two, and three letter words. Deciding that
a frequency table would shine some light on the nature of the
encryption, I wrote the following.

\begin{verbatim}
from collections import defaultdict
def frequency_table(s):
    """Gives the frequency table for characters in an input string"""
    length = len(s)
    f_table = defaultdict(lambda: 0)
    for character in s:
        f_table[character] += 1
    return {k:v/length for (k,v) in f_table.items()} # Won't work in python 2.x!
\end{verbatim}

This code was to produce the frequency table of characters found in
the input text. I followed with a helper function to sort the table on
the frequency, and applied the code to the ciphertext itself. The
results are shown in Table 1. (From now on, lowercase letters are
ciphertext and uppercase letters denote plaintext.)

\begin{table}[htb]
\caption{Frequency table for ciphertext}
\centering
\begin{tabular}{lr}
 & \\
r & 0.11394781302797072\\
g & 0.07828045804392716\\
n & 0.07809273512295851\\
b & 0.07584006007133472\\
a & 0.06269945560352919\\
v & 0.06251173268256054\\
u & 0.05819410550028158\\
f & 0.056129153369626435\\
e & 0.05519053876478318\\
q & 0.03679369250985545\\
z & 0.03379012577435705\\
y & 0.03341467993241975\\
h & 0.02703210061948564\\
\n & 0.026468931856579687\\
p & 0.024216256804955885\\
l & 0.023465365121081282\\
s & 0.01989862962267693\\
j & 0.01952318378073963\\
, & 0.016519617045241224\\
o & 0.016144171203303925\\
t & 0.015393279519429322\\
c & 0.014830110756523372\\
. & 0.011451098179087666\\
i & 0.011451098179087666\\
x & 0.007696639759714661\\
“ & 0.004880795945184907\\
” & 0.004880795945184907\\
- & 0.0022526750516238033\\
d & 0.0016895062887178525\\
w & 0.0015017833677492022\\
k & 0.0011263375258119017\\
’ & 0.0009386146048432514\\
* & 0.0009386146048432514\\
m & 0.0007508916838746011\\
? & 0.0005631687629059508\\
! & 0.0005631687629059508\\
— & 0.00037544584193730055\\
\_ & 0.00037544584193730055\\
; & 0.00018772292096865028\\
\end{tabular}
\end{table}

\section{Frequency Analysis on Substrings}
\label{sec-2}
Immediately noticeable is the high probability of the letter 'r', with
almost ten percent. I then wanted to consider some of the repeated
three letter words I had seen. I generalized the single-character
frequency table to handle any substrings of length \verb~n~. After removing
spaces from the ciphertext, I obtained the table for substrings of
length three (Table 2).

\begin{verbatim}
def n_frequency_table(s, n):
    """Gives a general frequency table of substrings of length n"""
    length = len(s)
    f_table = defaultdict(lambda:0)
    for i in range(length + 1 - n):
        f_table[s[i:i+n]] += 1
    return {k:v/length for (k,v) in f_table.items()}
\end{verbatim}

\begin{table}[htb]
\caption{Frequency table of length-3 substrings}
\centering
\begin{tabular}{lr}
 & \\
gur & 0.01576872536136662\\
naq & 0.0071334709968087105\\
vat & 0.006570302233902759\\
lbh & 0.005631687629059508\\
gun & 0.005256241787122208\\
uvf & 0.004505350103247607\\
ung & 0.004505350103247607\\
nir & 0.004317627182278956\\
ure & 0.004317627182278956\\
rag & 0.003942181340341655\\
ire & 0.0035667354984043552\\
\n\n“ & 0.0035667354984043552\\
uni & 0.003379012577435705\\
abg & 0.003379012577435705\\
vba & 0.0031912896564670547\\
rfg & 0.0030035667354984044\\
,na & 0.002815843814529754\\
”\n\n & 0.002815843814529754\\
zna & 0.002815843814529754\\
rer & 0.002815843814529754\\
.”\n & 0.002628120893561104\\
qgu & 0.002628120893561104\\
\ldots{} & \ldots{}\\
\end{tabular}
\end{table}

Noting the frequency of the letters 'g' and 'r' and the three letter
word 'gur', I began assuming that 'gur' translated to 'THE' in
plaintext. I also saw that 'n' was the most common single letter word,
and so assumed it to correspond to 'A'. With these letters translated,
I noticed some other likely words, such as 'j' corresponding to 'W' to
create the word 'WHAT', and 'e' corresponding to 'R' to create
'THREE'. After some such substitution, I ran the code on the text to
obtain a partial translation; an excerpt follows below.

\begin{verbatim}
from re import sub
with open("./HW-3.txt.e") as f:
    st = f.read().lower()

st2 = sub("j", "W", sub("y", "L", sub("e", "R", sub("n", "A", sub("g", "T", st)))))
sub("u", "H", sub("r", "E", sub("f", "S", sub("v", "I", st2))))
\end{verbatim}

\begin{verbatim}
“WHAT A WbzAa—bH, WHAT A WbzAa!” pRIEq THE xIat bs obHEzIA, WHEa WE
HAq ALL THREE REAq THIS EcISTLE. “qIq I abT TELL lbh HbW dhIpx Aaq
RESbLhTE SHE WAS? WbhLq SHE abT HAiE zAqE Aa AqzIRAoLE dhEEa? IS IT abT
A cITl THAT SHE WAS abT ba zl LEiEL?”

“sRbz WHAT I HAiE SEEa bs THE LAql SHE SEEzS IaqEEq Tb oE ba A iERl
qIssEREaT LEiEL Tb lbhR zAwESTl,” SAIq HbLzES, pbLqLl. “I Az SbRRl
THAT I HAiE abT oEEa AoLE Tb oRIat lbhR zAwESTl’S ohSIaESS Tb A zbRE
ShppESSshL pbapLhSIba.”
\end{verbatim}

\section{Further Substitutions}
\label{sec-3}
At this point, it was clear that this ciphertext was indeed
monoalphabetic. So, I continued to utilize knowledge of common words
in English to complete the cipher. Upon finding the first five to ten
letters, the rest came very easily. The code used to fully translate
the text follows; \verb~reverse_cipher~ is a dict mapping ciphertext
letters to plaintext letters. The full decrypted text is found in the
file \verb~HW-3.txt~.

\begin{verbatim}
strlist = []
for char in st:
    if char in reverse_cipher:
        strlist.append(reverse_cipher[char])
    else:
        strlist.append(char)

''.join(strlist)
\end{verbatim}

\begin{table}[htb]
\caption{Cipher used}
\centering
\begin{tabular}{ll}
 & \\
plaintext & ciphertext\\
A & n\\
B & o\\
C & p\\
D & q\\
E & r\\
F & s\\
G & t\\
H & u\\
I & v\\
J & w\\
K & x\\
L & y\\
M & z\\
N & a\\
O & b\\
P & c\\
Q & d\\
R & e\\
S & f\\
T & g\\
U & h\\
V & i\\
W & j\\
X & k\\
Y & l\\
Z & m\\
\end{tabular}
\end{table}
% Emacs 25.1.1 (Org mode 8.2.10)
\end{document}